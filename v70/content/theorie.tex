\section{Zielsetzung}
In der Vakuumtechnik wird sich mit der Erzeugung, Aufrechterhaltung und Messung von einem Vakuum beschäftigt,
welche mittlerweile eine wichtige Rolle für die Industrie und Forschung eingenommen hat. Aus diesem Grunde wird
in diesem Versuch mit verschiedenen Theoretischen Begriffe und Geräten auseinandergesetzt. Zu diesem Zwecke werden
die Evakurierungskurven einer Drehschieberpumpe sowie einer Turbomolekularpumpe aufgenommen und eine Leckratenmessung 
durchgeführt.

\section{Theorie}
\label{sec:Theorie}

\subsection{Vakuum}
\label{sec:vakuumtheo}

Im allgemeinen wird sobald der Druck in einem Gefäß unterhalb des Umgebungsdruckes ist von einem Vakuum gesprochen. Normgerechter 
wird jedoch erst ab einem Druck von $p=\SI{300}{\milli\bar}$, was ein Wert unterhalb dem niedriegsten Atmosphärischendruck auf der 
Erde ist, über ein Vakuum gesprochen.\\
In diesem Versuch wird mit guter Näherung das Gas als ideales Gas angenommen. Die Folge hier raus ist das sämtliche Wechselwirkungen 
als vollkommende elastische Stöße geschehen und die Teilchen ohne Ausdehnung sind. Desweiteren führt uns dies zur idealen Glasgleichung:
\begin {equation}
 p\cdot V\,=\, N\cdot K_B\cdot T
\label{eqn:idealgas}
\end{equation}
in dieser stehen $p$ für den Druck, $V$ für das Volumen, $N$ die Teilchenzahl, $T$ die Temperatur und $k_B$ für die Boltzmann Konstante. 
Als Spezialfall existiert das Gesetz von Boyle-Mariotte in diesem gilt bei konatanter Temperatur 
\begin {equation*}
 p\,\propto \, V^{-1}.
\end{equation*}
Zusätzlich ist zu erwähnen das der Druck als Kraft pro Fläche definiert ist somit gilt
\begin {equation*}
 \SI{1}{\pascal}\,=\,\SI{0.01}{\milli\bar}\,=\,\SI{1}{\newton\per\metre\squared}.
\end{equation*}
Mit Partialdruck hingegen wird der Druck einer Komponente eines Gasgemisches bezeichnet, folglich ist bei einem Gemisch aus 
idealen Gasen die Summe aller Partialdrücke somit der Totaldruck. Zusätzlich zur Angabe des Druckes hat sich bei sehr geringen 
Drücken die Teilchenzahldichte als alternative Angabe bewährt. Ein Beispiel für solche Angaben ist Der Weltraum, hier wird die 
Anzahl an Teilchen in einem Raumbereich angegeben.
\subsubsection{Vakuumbereiche und Strömungen}
Da durch abnehmendnen Druck sich die Teilchenzahldichte in einem Volumen reduziert sind physikalisch mehrere Vakuumbereiche 
definiert. Bei abnehmender Teilchenzahldichte steigt die mittlere freie Weglänge, wodurch sich folglich die Anzahl an Stößen der 
Gasteilchen untereinander reduziert. Durch die ideale Gasgleichung aus \autoref{eqn:idealgas} lässt sich ableiten das durch eine 
Druckänderung zusätzlich eine Temperaturänderung stattfindet. Bei abnehmender Temperatur sinkt die Teilchengeschwindigkeit nach der 
Maxwell-Boltzmann Verteilung und somit wird unteranderm auch die Strömung durch einen Leiter bestimmt.\\
Im folgenden ist eine Auflistung der Relevanten Druckbereiche mit den vorherschenden Strömungen.


\begin{itemize}
    \item Der erste Bereich der beim erzeugen eines Vakuums durchlaufen wird ist das Grobvakuum, von diesem wir din einem 
    Druckbereich von $\SI{300}{\milli\bar}$ bis $\SI{1}{\milli\bar}$. In diesem Bereich liegt die Viskose Strömung oder Kontinuumsströmung 
    vor welche nochmal in die laminare und turbulente Strömung unterteilt wird. Dies Geschieht mit hilfe der Reynoldszahl.

    \item Das Feinvakuum, der nächste Bereich welcher sich zwischen $\SI{1}{\milli\bar}$ und $\SI{e-3}{\milli\bar}$ befindet, 
    besitzt als Strömungsart die Knudsen-Strömung. Knudsenströmung beschreibt Strömungen welche sich zwischen der Viskosen und der 
    Molekularen Strömung.

    \item Ab $\SI{e-3}{\milli\bar}$ wird von einem Hochvakuum gesprochen, dieser bereich wird nochmal in Ultrahoch ab $\SI{e-8}{\milli\bar}$ 
    und extrem hohes Vakuum ab $\SI{e-11}{\milli\bar}$ unterteilt. In diesem Bereich findet die Molekulare Strömung statt, bei dieser 
    Wechselwirken die Teilchen untereinander kaum noch da die mittlere freie Weglänge größer als der Durchmesser des Strömungskanals ist.
    
\end{itemize}

Die soeben erwähnten Strömungsarten haben große Relevanz bei Praktischen Anwendungen, wie zum Beispiel das effektive Saugvermögen 
von Vakuumpumpen. Neben der bereits erwähnten Reynoldszahl ist die Knudsenzahl massgeblich für die Unterteilung der Strömungsart.
Die Knudsenzahl wird über die Mittlere freie Weglänge $\bar l$ und die Weite des Strömungskanals $d$ als
\begin {equation}
 \frac{\bar l}{d}
\label{eqn:knudsen}
\end{equation}
definiert.\\

\subsection{Vakuumerzeugung}


\subsection{Messinstrumente}