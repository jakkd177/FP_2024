\section{Zielsetzung}
In der Vakuumtechnik wird sich mit der Erzeugung, Aufrechterhaltung und Messung von einem Vakuum beschäftigt,
welche mittlerweile eine wichtige Rolle für die Industrie und Forschung eingenommen hat. Aus diesem Grunde wird
in diesem Versuch mit verschiedene Theoretische Begriffe und Geräte auseinandergesetzt. Zu diesem Zwecke werden
die Evakurierungskurven einer Drehschieberpumpe sowie einer Turbomolekularpumpe aufgenommen und eine Leckratenmessung 
durchgeführt.

\section{Theorie}
\label{sec:Theorie}

\subsection{Vakuum}
\label{sec:vakuumtheo}

Im allgemeinen wird sobald der Druck in einem Gefäß unterhalb des Umgebungsdruckes ist von einem Vakuum gesprochen. Normgerechter 
wird jedoch erst ab einem Druck von $p=\SI{300}{\milli\bar}$, was ein Wert unterhalb dem niedriegsten Atmosphärischendruck auf der 
Erde ist, über ein Vakuum gesprochen.\\
In diesem Versuch wird mit guter Näherung das Gas als ideales Gas angenommen. Die Folge hier raus ist das sämtliche Wechselwirkungen 
als vollkommende elastische Stöße geschehen und die Teilchen ohne Ausdehnung sind. Desweiteren führt uns dies zur idealen Glasgleichung:
\begin {equation}
pV=
\label{eqn:idealgas}
\end{equation}