\section{Zielsetzung}
In der Vakuumtechnik wird sich mit der Erzeugung, Aufrechterhaltung und Messung von einem Vakuum beschäftigt,
welche mittlerweile eine wichtige Rolle für die Industrie und Forschung eingenommen hat. Aus diesem Grunde wird
in diesem Versuch mit verschiedenen Theoretischen Begriffe und Geräten auseinandergesetzt. Zu diesem Zwecke werden
die Evakuierungskurven einer Drehschieberpumpe sowie einer Turbomolekularpumpe aufgenommen und eine Leckratenmessung 
durchgeführt.

\section{Theorie}
\label{sec:Theorie}

\subsection{Fachbegriffe}
Im Verlauf dies Protokolls werden einige Fachbegriffe verwendet diese sollen nun kurz erörtert werden.
\begin{itemize}
    \item Mittlere freie Weglänge: \\
    Beschreibt die Durchschnittliche Strecke die ein Teilchen zurück legt bevor es mit einem anderen kollidiert oder anderbeitig 
    wechselwirkt.

    \item Sorptionen: \\
    Unter diesem Begriff werden mehrere Prozesse zusammen gefasst die eine Anreicherung eines Stoffes zur folge haben. In diesem 
    Versuch sind die Adsorption, das Anlagern auf Grenzflächen beziehungsweise Oberflächen, die Absorption, das aufnehmen eines 
    Stoffes innerhalb einer Phase und die Desorption, welcher grundsätlich einen gegenteiligen Prozess der Adsorption darstellt, die wichtigsten.

    \item Lecks: \\
    Hier wird zwischen virtuellen und realen unterschieden, wobei die realen von aussen messbar sind und die virtuellen aufgrund 
    von Desorptione und Materialleinschlüssen entstehen und beim Bau der Anlagen verhindert oder reduziert werden müssen.

    \item Saugvermögen:\\
    Beschreibt wie viel Volumen pro Zeiteinheit durch eine Pumpe aus dem Rezipienten gepumpt werden kann. Der theoretische Wert kann 
    hierbei vom realen Saugvermögen abweichen hier für gibt es mehrere mögliche Gründe.
\end{itemize}

\subsection{Vakuum}
\label{sec:vakuumtheo}

Im allgemeinen wird sobald der Druck in einem Gefäß unterhalb des Umgebungsdruckes ist von einem Vakuum gesprochen. Normgerechter 
wird jedoch erst ab einem Druck von $p=\SI{300}{\milli\bar}$, was ein Wert unterhalb dem niedriegsten Atmosphärischendruck auf der 
Erde ist, über ein Vakuum gesprochen.\\
In diesem Versuch wird mit guter Näherung das Gas als ideales Gas angenommen. Die Folge hier raus ist das sämtliche Wechselwirkungen 
als vollkommende elastische Stöße geschehen und die Teilchen ohne Ausdehnung sind. Desweiteren führt uns dies zur idealen Glasgleichung:
\begin {equation}
 p\cdot V\,=\, N\cdot K_B\cdot T
\label{eqn:idealgas}
\end{equation}
in dieser stehen $p$ für den Druck, $V$ für das Volumen, $N$ die Teilchenzahl, $T$ die Temperatur und $k_B$ für die Boltzmann Konstante. 
Als Spezialfall existiert das Gesetz von Boyle-Mariotte in diesem gilt bei konatanter Temperatur 
\begin {equation*}
 p\,\propto \, V^{-1}.
\end{equation*}
Zusätzlich ist zu erwähnen das der Druck als Kraft pro Fläche definiert ist somit gilt
\begin {equation*}
 \SI{1}{\pascal}\,=\,\SI{0.01}{\milli\bar}\,=\,\SI{1}{\newton\per\metre\squared}.
\end{equation*}
Mit Partialdruck hingegen wird der Druck einer Komponente eines Gasgemisches bezeichnet, folglich ist bei einem Gemisch aus 
idealen Gasen die Summe aller Partialdrücke somit der Totaldruck. Zusätzlich zur Angabe des Druckes hat sich bei sehr geringen 
Drücken die Teilchenzahldichte als alternative Angabe bewährt. Ein Beispiel für solche Angaben ist Der Weltraum, hier wird die 
Anzahl an Teilchen in einem Raumbereich angegeben.
\subsubsection{Vakuumbereiche und Strömungen}
Da durch abnehmendnen Druck sich die Teilchenzahldichte in einem Volumen reduziert sind physikalisch mehrere Vakuumbereiche 
definiert. Bei abnehmender Teilchenzahldichte steigt die mittlere freie Weglänge, wodurch sich folglich die Anzahl an Stößen der 
Gasteilchen untereinander reduziert. Durch die ideale Gasgleichung aus \autoref{eqn:idealgas} lässt sich ableiten das durch eine 
Druckänderung zusätzlich eine Temperaturänderung stattfindet. Bei abnehmender Temperatur sinkt die Teilchengeschwindigkeit nach der 
Maxwell-Boltzmann Verteilung und somit wird unteranderm auch die Strömung durch einen Leiter bestimmt.\\
Im folgenden ist eine Auflistung der Relevanten Druckbereiche mit den vorherschenden Strömungen.


\begin{itemize}
    \item Der erste Bereich der beim erzeugen eines Vakuums durchlaufen wird ist das Grobvakuum, von diesem wir din einem 
    Druckbereich von $\SI{300}{\milli\bar}$ bis $\SI{1}{\milli\bar}$. In diesem Bereich liegt die Viskose Strömung oder Kontinuumsströmung 
    vor welche nochmal in die laminare und turbulente Strömung unterteilt wird. Dies Geschieht mit hilfe der Reynoldszahl.

    \item Das Feinvakuum, der nächste Bereich welcher sich zwischen $\SI{1}{\milli\bar}$ und $\SI{e-3}{\milli\bar}$ befindet, 
    besitzt als Strömungsart die Knudsen-Strömung. Knudsenströmung beschreibt Strömungen welche sich zwischen der Viskosen und der 
    Molekularen Strömung.

    \item Ab $\SI{e-3}{\milli\bar}$ wird von einem Hochvakuum gesprochen, dieser bereich wird nochmal in Ultrahoch ab $\SI{e-8}{\milli\bar}$ 
    und extrem hohes Vakuum ab $\SI{e-11}{\milli\bar}$ unterteilt. In diesem Bereich findet die Molekulare Strömung statt, bei dieser 
    Wechselwirken die Teilchen untereinander kaum noch da die mittlere freie Weglänge größer als der Durchmesser des Strömungskanals ist.
    
\end{itemize}

Die soeben erwähnten Strömungsarten haben große Relevanz bei Praktischen Anwendungen, wie zum Beispiel das effektive Saugvermögen 
von Vakuumpumpen. Neben der bereits erwähnten Reynoldszahl ist die Knudsenzahl massgeblich für die Unterteilung der Strömungsart.
Die Knudsenzahl wird über die Mittlere freie Weglänge $\bar l$ und die Weite des Strömungskanals $d$ als
\begin {equation}
 \frac{\bar l}{d}
\label{eqn:knudsen}
\end{equation}
definiert.
\begin{figure}[h]
    \centering
    \includegraphics[width=.4\linewidth]{"content/images/Strömungen.png"}
    \caption{Schematische Darstellung der Strömungsverhalten. Blaue Pfeile stellen geringe mittlere freie Weglängen dar und grüne lange. \protect \cite{pfeiffer} }
\label{fig:stroemung}
\end{figure}
In \autoref{fig:stroemung} sind noch einmal die Strömungsarten schematisch dargestellt. Zusehen ist hier das kleine Leiterdurchmesser 
bei großen mittleren freien Weglängen zu chaotischen flussverhalten führen, folglich sind beim Grobvakuum kleine Leiterdurchmesser 
ausreichend um eine hohe Flussrate zu erreichen und je besser das Vakuum wird desto größer muss der Leiter werden um hohe Flussraten 
zu erreichen.

\subsection{Vakuumerzeugung}
Vakuumpumpen unterscheiden sich je nach Arbeitsberecih teils gravierend 
Deswegen werden Vakuuumpumpen, welche zur Erzeugung der Vakua verwendet werden, grundlegend in die Gasbindenden und die Gasfördernden unterteilt. 
Gasfördernde Pumpen wiederum sind in kinetische und verdrängende unterteilt. Die beiden letzteren Typen sind in diesem Versuch verwendet worden 
und werden später noch einmal genauer diskutiert.
\begin{itemize}
    \item Verdrängende Vakuumpumpen:\\
    Bei diesem Pumpentyp wird mittels Veränderung des Volumina unter Verwendung des Boyle-Mariotte Gesetz eine Veränderung des Druckes 
    verursacht. Aus disem Grunde werden die Teilchen beim ausgleichen der Drücke in die Pumpenkammer beziehungsweise nach der Kammer 
    in die Umgebung abgegeben. Die Drehschieberpumpe ist ein Vertreter dieses Pumpentyps.


    \item Kinetische Vakuumpumpen:\\
    Bei diesem Pumpentyp werden die Teilchen in Pumprichtung beschleunigt, dies geschieht durch eine ausgeklügelten Aufbau von Rotoren 
    und Statoren wodurch die Vorzugsrichtung der Gasteilchen aus dem Volumen und der Pumpe raus ist. Wichtig für diesen Pumpentyp 
    ist eine längere mittlere freie Weglänge als die Abstände der Wechselkirkungspunkte in der Pumpe, was eine Verwendung unter normal 
    Druck nicht funktionabel macht. Ein Beispiel ist für diesen Pumpentyp ist die Turbomolekularpumpe welche in diesem Versuch zum einsatz kommt.

    \item Gasbindende Vakuumpumpe:\\
    Bei diesem Typ wird sich die Sorption zu nutzen gemacht, es werden Gasteilchen an die Oberfläche der Pumpe gebunden. Ein Beispiel 
    hierfür ist die Titaniumsublimationspumpe welche Titan auf eine gekühlte Adsorptionfläche aufdampft um an diesem die Restgase chemisch 
    zu binden welche zufällig dort auftreffen. Ist das aufgedampfte Titan gesättigt muss erneut welches aufgedampft werden um weiterhin ein binden 
    des Gases zu ermöglichen. Bei diesem Verfahren sind bereits hohe Güten von Vakua von nöten um eine Effektive Nutzung zu ermöglichen.
\end{itemize}

\subsection{Messinstrumente}
Zum vermessen der Drücke im Rezipienten werden unterschiedliche Messgeräte verwendet, diese haben auch unterschiedliche Arbeitbereiche 
in denen die Instrumente effektiv arbeiten. Im folgenden sind einige Messgeräte kurz aufgeführt und die funktionsweise erklärt.
\begin{itemize}
 \item Pirani-Vakuummeter:\\
 Dieses Vakuummeter wird optimaler Weise fürs Feinvakuum genutzt, da in diesem Bereich die Wärmeleitfähigkeit proportional zum Druck ist.
Gemessen wird hierbei die Veränderung des Wiederstandes eines Glühdrahtes welcher im Rezipienten ist und mit einem konstanten Strom beaufschlagt 
wird. Zur Messung des Wiederstandes wird letztendlich eine Wheatstone-Brücke verwendet und mit diesem eine Temperatur ermittelt welche wiederum 
auf den Druck im Rezipienten zurück geführt wird.
 \item Penning-Vakuummeter:\\
 Hierbei wird auch von einem Kalthkathoden-Ionisationsvakuumeter gesprochen. Die zur Ionisation verwendeten Elektronen werden hier mittels 
 elektrischen Feld frei. Gemessen wird der Entladungsstrom, sprich die Ionisierten Atome welche zur Kathode beschleunigt werden.
 \item Bayard-Alpert-Vakuummeter:\\
 Dieser Typ wird auch als Glükathoden-Ionisationsvakuumeter bezeichnet.
 Die arbeitsweis ist analog zum Penning-Vakuummeter, jedoch werden hier die Elektronen mittels thermischer Emission Freigesetzt.
 \item Piezo-Vakuummeter:\\
 Die abgabe einer Spannung wenn eine Piezo-Kristall einer Kraft ausgesetzt wird ist in diesem Vakuumeter der sich zur Nutzengemachte 
 Vorgang. Mittels der so gemssen Spannung kann auf den Druck im System zurückgeschlossen werden. Aufgrund der ähnlichen Arbeitsbereiche werden Piezo- sowie Pirani-Vakuummeter meist in einem Messgerät verbaut 
\end{itemize}

\subsection{Bestimmung des Saugvermögens}
Bei der Bestimmung des Saugvermögens ist der Leitwert zu beachten, durch diesen wird das theoretische Saugvermögen $S_0$
vermindert und gibt den reziproken Strömungswiederstand an. Somit ergibt sich das effiktive Saugvermögen als 
\begin {equation}
 S_{eff}\,=\, \frac{S_0\cdot L}{S_0+L}.
\label{eqn:effsaug}
\end{equation}
Das Saugvermögen kann auf unterschiedliche Arten bestimmt werden in diesem Versuch geschieht dies auf zwei Arten, welche nun 
kurz beschrieben werden.
\subsubsection{Die Evakuierungskurve}
Augehend von einem idealen Gas und der \autoref{eqn:startherleit} wird nun das effektive Saugvermögen im Falle der Evakuierungskurve hergeleitet.
\begin{equation}
    p\cdot V\,=\, \text{const}
    \label{eqn:startherleit}
\end{equation}
Durch einmal zeitliches ableit und multiplizieren mit $p$ kann die zeitliche Änderung des Volumens als Saugvermögen $S$ definiert werden.
\begin{equation}
\frac{\text{d}V}{\text{d}t} = S = - \frac{V}{p} \frac{\text{d}p}{\text{d}t}
\end{equation}
Die so erhaltene Differentialgleichung wird mittels Exponentialansatz gelöst, hierzu werden im Exponenten 
\subsubsection{Die Leckratenmessung}

